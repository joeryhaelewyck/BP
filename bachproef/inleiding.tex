%%=============================================================================
%% Inleiding
%%=============================================================================

\chapter{\IfLanguageName{dutch}{Inleiding}{Introduction}}
\label{ch:inleiding}

Microservices is een architectuur die steeds populairder wordt. Bedrijven beginnen steeds meer over te schakelen naar deze structuur. Bestaande ondernemingen vertrekken meestal vanuit een monolithische architectuur. De transformatie van een monoliet naar microservices is een proces vol uitdagingen. Er zijn verschillende manieren om deze verandering tot stand te brengen. De focus van dit onderzoek ligt echter op de impact die een dergelijke transformatie met zich meebrengt.\\

De monolitische architectuur in software ontwikkeling betekent dat de programma's, applicaties, \emph{deployment}, interface, ... 1 geheel vormen. Voor startende ondernemingen en hoofdzakelijk kleine bedrijven is deze architectuur vaak voldoende om aan de noden van het bedrijf te voldoen.\\

Een groot nadeel van de monolieten is de schaalbaarheid. Het is moeilijk om individuele componenten uit te breiden. Hoe groter de monoliet, hoe complexer het systeem. Het wordt moeilijker om aanpassingen door te voeren, de instaptijd van nieuwe ontwikkelaars wordt langer en het zorgt ervoor dat het bedrijf minder flexibel kan reageren op marktsveranderingen.\\

De microservice architectuur biedt een oplossing voor deze problemen, waardoor bedrijven geneigd zijn om de overschakeling te maken. Terwijl een monolithische applicatie een enkele verenigde eenheid is, zal een applicatie bestaande uit microservices het geheel opsplitsen in een verzameling van kleinere onafhankelijke eenheden. Deze architectuur zorgt voor meer flexibiliteit omdat de componenten onafhankelijk \emph{gedeployed} kunnen worden. Het zorgt er ook voor dat de applicaties makkelijker uit te breiden zijn. \\

Naast alle voordelen die microservices biedt, brengt het ook wat uitdagingen met zich mee. Het systeem wordt complexer en mits alle componenten onafhankelijk werken, zorg dit voor een groter druk op de \emph{load balancer}.

\section{\IfLanguageName{dutch}{Probleemstelling}{Problem Statement}}
\label{sec:probleemstelling}

De belangrijkste doelgroep van dit onderzoek, zijn de bedrijven die willen overschakelen van een monolitische architectuur naar microservices. Het onderzoek kan ook een meerwaarde hebben voor personen die meer kennis willen verzamelen omtrent deze architecturen. 

\section{\IfLanguageName{dutch}{Onderzoeksvraag}{Research question}}
\label{sec:onderzoeksvraag}

Het doel van dit onderzoek is om te weten te komen wat de precieze impact is wanneer een bedrijf overschakelt van een monolitische architectuur naar een architectuur bestaande uit microservices.

De 3 hoofdvragen van dit onderzoek zijn als volgt:
\begin{itemize}
    \item Wat zijn de gevolgen van de transformatie?
    \item Wat zijn de financiële kosten die gemaakt worden tijdens de transformatie?
    \item Welk effect heeft deze transformatie op het sociale aspect van het bedrijf?
\end{itemize}

\section{\IfLanguageName{dutch}{Onderzoeksdoelstelling}{Research objective}}
\label{sec:onderzoeksdoelstelling}

Gebaseerd op de onderzoeksvragen die terug te vinden zijn in de vorige paragraaf. Wordt het volgende resultaat verwacht:

\begin{itemize}
    \item Een literatuurstudie omtrent microservices en de monolitische architectuur in software ontwikkeling. Waarin de voordelen en uitdagingen aangehaald worden.
    \item Een vergelijking tussen de twee architecturen. 
    \item Een model die als basis dient om een schatting te maken van de impact.
    \item Het model toepassen op een hedendaagse bedrijfssituatie.
\end{itemize}
Naast alle voordelen die microservices biedt, brengt het ook wat uitdagingen met zich mee. Het systeem wordt complexer en mits alle componenten onafhankelijk werken, zorg dit voor een groter druk op de \emph{load balancer}.

\section{\IfLanguageName{dutch}{Probleemstelling}{Problem Statement}}
\label{sec:probleemstelling}

De belangrijkste doelgroep van dit onderzoek, zijn de bedrijven die willen overschakelen van een monolithische architectuur naar microservices. Het onderzoek kan ook een meerwaarde hebben voor personen die meer kennis willen verzamelen omtrent deze architecturen. 

\section{\IfLanguageName{dutch}{Onderzoeksvraag}{Research question}}
\label{sec:onderzoeksvraag}

Het doel van dit onderzoek is om te weten te komen wat de precieze impact is wanneer een bedrijf overschakelt van een monolithische architectuur naar een architectuur bestaande uit microservices.

De 3 hoofdvragen van dit onderzoek zijn als volgt:
\begin{itemize}
    \item Wat zijn de gevolgen van de transformatie?
    \item Wat zijn de financiële kosten die gemaakt worden tijdens de transformatie?
    \item Welk effect heeft deze transformatie op het sociale aspect van het bedrijf?
\end{itemize}

\section{\IfLanguageName{dutch}{Onderzoeksdoelstelling}{Research objective}}
\label{sec:onderzoeksdoelstelling}

Gebaseerd op de onderzoeksvragen die terug te vinden zijn in de vorige paragraaf. Wordt het volgende resultaat verwacht:

\begin{itemize}
    \item Een literatuurstudie omtrent microservices en de monolithische architectuur in software ontwikkeling. Waarin de voordelen en uitdagingen aangehaald worden.
    \item Een vergelijking tussen de twee architecturen. 
    \item Een model die als basis dient om een schatting te maken van de impact.
    \item Het model toepassen op een hedendaagse bedrijfssituatie.
\end{itemize}

\section{\IfLanguageName{dutch}{Opzet van deze bachelorproef}{Structure of this bachelor thesis}}
\label{sec:opzet-bachelorproef}

% Het is gebruikelijk aan het einde van de inleiding een overzicht te
% geven van de opbouw van de rest van de tekst. Deze sectie bevat al een aanzet
% die je kan aanvullen/aanpassen in functie van je eigen tekst.

De rest van deze bachelorproef is als volgt opgebouwd:

In Hoofdstuk 2 wordt een over--zicht gegeven van de stand van zaken binnen het onderzoeksdomein, op basis van een literatuurstudie.

In Hoofdstuk \ref{ch:methodologie} wordt de methodologie toegelicht en worden de gebruikte onderzoekstechnieken besproken om een antwoord te kunnen formuleren op de onderzoeksvragen.

In Hoofdstuk \ref{ch:model} wordt het model opgebouwd die als basisstuk dient om de schatting te maken van de impact van de transformatie.

In Hoofdstuk \ref{ch:usecase} passen we het model toe op een bedrijf die momenteel een monolithische architectuur gebruikt en overschakelt naar een architectuur bestaande uit microservices.
--
\section{\IfLanguageName{dutch}{Opzet van deze bachelorproef}{Structure of this bachelor thesis}}
\label{sec:opzet-bachelorproef}

% Het is gebruikelijk aan het einde van de inleiding een overzicht te
% geven van de opbouw van de rest van de tekst. Deze sectie bevat al een aanzet
% die je kan aanvullen/aanpassen in functie van je eigen tekst.

De rest van deze bachelorproef is als volgt opgebouwd:

In Hoofdstuk~\ref{ch:stand-van-zaken} wordt een over--zicht gegeven van de stand van zaken binnen het onderzoeksdomein, op basis van een literatuurstudie.

In Hoofdstuk~\ref{ch:methodologie} wordt de methodologie toegelicht en worden de gebruikte onderzoekstechnieken besproken om een antwoord te kunnen formuleren op de onderzoeksvragen.

In Hoofdstuk \ref{ch:model} wordt het model opgebouwd die als basisstuk dient om de schatting te maken van de impact van de transformatie.

In Hoofdstuk \ref{ch:usecase} passen we het model toe op een bedrijf die momenteel een monolitische architectuur gebruikt en overschakelt naar een architectuur bestaande uit microservices.

% TODO: Vul hier aan voor je eigen hoofstukken, één of twee zinnen per hoofdstuk

In Hoofdstuk~\ref{ch:conclusie}, tenslotte, wordt de conclusie gegeven en een antwoord geformuleerd op de onderzoeksvragen. Daarbij wordt ook een aanzet gegeven voor toekomstig onderzoek binnen dit domein.