%%=============================================================================
%% Inleiding
%%=============================================================================

\chapter{\IfLanguageName{dutch}{Inleiding}{Introduction}}
\label{ch:inleiding}

Microservices is een architectuur die steeds populairder wordt. Bedrijven beginnen steeds meer over te schakelen naar deze structuur. Bestaande ondernemingen vertrekken meestal vanuit een monolithische architectuur. De transformatie van een monoliet naar microservices is een proces vol uitdagingen. Er zijn verschillende manieren om deze verandering tot stand te brengen. De focus van dit onderzoek ligt echter op de impact die een dergelijke transformatie met zich meebrengt.\\

De monolitische architectuur in software ontwikkeling betekent dat de programma's, applicaties, \emph{deployment}, interface, ... 1 geheel vormen. Voor startende ondernemingen en hoofdzakelijk kleine bedrijven is deze architectuur vaak voldoende om aan de noden van het bedrijf te voldoen.\\

Een groot nadeel van de monolieten is de schaalbaarheid. Het is moeilijk om individuele componenten uit te breiden. Hoe groter de monoliet, hoe complexer het systeem. Het wordt moeilijker om aanpassingen door te voeren, de instaptijd van nieuwe ontwikkelaars wordt langer en het zorgt ervoor dat het bedrijf minder flexibel kan reageren op marktsveranderingen.\\

De microservice architectuur biedt een oplossing voor deze problemen, waardoor bedrijven geneigd zijn om de overschakeling te maken. Terwijl een monolithische applicatie een enkele verenigde eenheid is, zal een applicatie bestaande uit microservices het geheel opsplitsen in een verzameling van kleinere onafhankelijke eenheden. Deze architectuur zorgt voor meer flexibiliteit omdat de componenten onafhankelijk \emph{gedeployed} kunnen worden. Het zorgt er ook voor dat de applicaties makkelijker uit te breiden zijn. \\

Naast alle voordelen die microservices biedt, brengt het ook wat uitdagingen met zich mee. Het systeem wordt complexer en mits alle componenten onafhankelijk werken, zorg dit voor een groter druk op de \emph{load balancer}.

\section{\IfLanguageName{dutch}{Probleemstelling}{Problem Statement}}
\label{sec:probleemstelling}

Uit je probleemstelling moet duidelijk zijn dat je onderzoek een meerwaarde heeft voor een concrete doelgroep. De doelgroep moet goed gedefinieerd en afgelijnd zijn. Doelgroepen als ``bedrijven,'' ``KMO's,'' systeembeheerders, enz.~zijn nog te vaag. Als je een lijstje kan maken van de personen/organisaties die een meerwaarde zullen vinden in deze bachelorproef (dit is eigenlijk je steekproefkader), dan is dat een indicatie dat de doelgroep goed gedefinieerd is. Dit kan een enkel bedrijf zijn of zelfs één persoon (je co-promotor/opdrachtgever).

\section{\IfLanguageName{dutch}{Onderzoeksvraag}{Research question}}
\label{sec:onderzoeksvraag}

Wees zo concreet mogelijk bij het formuleren van je onderzoeksvraag. Een onderzoeksvraag is trouwens iets waar nog niemand op dit moment een antwoord heeft (voor zover je kan nagaan). Het opzoeken van bestaande informatie (bv. ``welke tools bestaan er voor deze toepassing?'') is dus geen onderzoeksvraag. Je kan de onderzoeksvraag verder specifiëren in deelvragen. Bv.~als je onderzoek gaat over performantiemetingen, dan 

\section{\IfLanguageName{dutch}{Onderzoeksdoelstelling}{Research objective}}
\label{sec:onderzoeksdoelstelling}

Wat is het beoogde resultaat van je bachelorproef? Wat zijn de criteria voor succes? Beschrijf die zo concreet mogelijk. Gaat het bv. om een proof-of-concept, een prototype, een verslag met aanbevelingen, een vergelijkende studie, enz.

\section{\IfLanguageName{dutch}{Opzet van deze bachelorproef}{Structure of this bachelor thesis}}
\label{sec:opzet-bachelorproef}

% Het is gebruikelijk aan het einde van de inleiding een overzicht te
% geven van de opbouw van de rest van de tekst. Deze sectie bevat al een aanzet
% die je kan aanvullen/aanpassen in functie van je eigen tekst.

De rest van deze bachelorproef is als volgt opgebouwd:

In Hoofdstuk~\ref{ch:stand-van-zaken} wordt een overzicht gegeven van de stand van zaken binnen het onderzoeksdomein, op basis van een literatuurstudie.

In Hoofdstuk~\ref{ch:methodologie} wordt de methodologie toegelicht en worden de gebruikte onderzoekstechnieken besproken om een antwoord te kunnen formuleren op de onderzoeksvragen.

% TODO: Vul hier aan voor je eigen hoofstukken, één of twee zinnen per hoofdstuk

In Hoofdstuk~\ref{ch:conclusie}, tenslotte, wordt de conclusie gegeven en een antwoord geformuleerd op de onderzoeksvragen. Daarbij wordt ook een aanzet gegeven voor toekomstig onderzoek binnen dit domein.