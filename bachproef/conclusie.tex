%%=============================================================================
%% Conclusie
%%=============================================================================

\chapter{Conclusie}
\label{ch:conclusie}

% TODO: Trek een duidelijke conclusie, in de vorm van een antwoord op de
% onderzoeksvra(a)g(en). Wat was jouw bijdrage aan het onderzoeksdomein en
% hoe biedt dit meerwaarde aan het vakgebied/doelgroep? 
% Reflecteer kritisch over het resultaat. In Engelse teksten wordt deze sectie
% ``Discussion'' genoemd. Had je deze uitkomst verwacht? Zijn er zaken die nog
% niet duidelijk zijn?
% Heeft het onderzoek geleid tot nieuwe vragen die uitnodigen tot verder 
%onderzoek?

Uit het onderzoek blijkt dat er heel wat uitdagingen zijn bij een transformatie van een monolithische architectuur naar een microservice-architectuur. Microservices bieden heel wat voordelen, maar de complexiteit zorgt er voor dat niet elke implementatie hiervan een succesverhaal zal zijn. Een bedrijf die afstapt van hun monolithische architectuur zal moeten investeren in nieuwe infrastructuur, bestaand personeel opleiden, nieuw personeel aanwerven en eventueel nieuwe licenties moeten aanschaffen. Dit is financieel een heel grote investering. Deze investering is ook geen garantie op succes. Een verkeerde implementatie of foute aanpak kan ervoor zorgen dat het systeem onnodige complexer wordt en de voordelen van microservices niet kan utiliseren. Een goed modulair ontwerp voor de start van de transformatie is onmisbaar. 

Naast de financiële impact, houden bedrijven best ook rekening met verloop van personeel. Het is moeilijk om iedereen mee te krijgen in een nieuw verhaal en dit is ook het geval voor een dergelijke transformatie.

De kleiner bedrijven hebben minder nood aan de voordelen van microservices. Zolang de monoliet onder controle blijft en de marktspositie van het bedrijf niet in gedrang komt, zijn er geen grote redenen om de overstap te maken. Als de monoliet alle noden van een bedrijf vervult, is er geen drang om over te schakelen naar een architectuur bestaande uit microservices. Als een bedrijf groeit en op het punt komt waar de monoliet de groei vertraagt, kan er met behulp van \emph{design patterns} een omschakeling opgestart worden naar microservices. De \emph{Strangler pattern} is een populaire strategie hiervoor.

Een kanttekening voor dit onderzoek is dat het niet verplicht is om te vertrekken vanuit een monolithische architectuur. Als het bedrijf er niet in slaagt om een goed gestructureerde monoliet op te zetten, zal het ook niet slagen in het uit bouwen van een architectuur met microservices.  ~\autocite{Tilkov2015}

Een goede voorbereiding is een \emph{must} om een goede transformatie te kunnen uitvoeren. Bedrijven moeten zeker zijn dat microservices helpen met het invullen van hun bedrijfsnoden. Er moet een onderscheid gemaakt worden tussen microservices en zakelijke functies/services. Ondernemingen moeten iedereen betrekken in het verhaal van de microservices. Hoofdzakelijk zal elk team zelfstandig werken en verantwoordelijk zijn voor 1 of meerdere microservices. Het bedrijf bouwt dus best hun teams rond deze microservices. 

De precieze impact kan niet worden voorspelt en zal voor elk bedrijf anders ingevuld worden. Dit onderzoek kan als hulpstuk gebruikt worden om de impact te bepalen.





 
