%%=============================================================================
%% Methodologie
%%=============================================================================

\chapter{\IfLanguageName{dutch}{Methodologie}{Methodology}}
\label{ch:methodologie}

%% TODO: Hoe ben je te werk gegaan? Verdeel je onderzoek in grote fasen, en
%% licht in elke fase toe welke stappen je gevolgd hebt. Verantwoord waarom je
%% op deze manier te werk gegaan bent. Je moet kunnen aantonen dat je de best
%% mogelijke manier toegepast hebt om een antwoord te vinden op de
%% onderzoeksvraag.

\section{Inleiding}

Het onderzoek is onderverdeeld in twee delen:
\begin{itemize}
    \item Het theoretisch onderzoek
    \item Toepassing op een \emph{use case} 
\end{itemize}

\section{Theoretisch onderzoek}

Het eerste deel van dit onderzoek vloeit voort uit de uitgevoerde literatuurstudie. De verzamelde kennis geeft een beter beeld op het onderzoeksdomein. Dit werd gedaan in verschillende fases:

\begin{itemize}
    \item Fase 1 : Informatie verzamelen omtrent de monolotische en microservicearchitectuur, zodat er basis gevormd wordt voor de volgende stappen van het onderzoeksproces. 
    \item Fase 2 : Dieper ingaan op de voordelen en uitdagingen van de microservicearchitectuur.
    \item Fase 3 : Een model opstellen waarbij er aangehaald wordt welke aspecten het meest geimpacteerd worden bij een transformatie van een monolitische naar een microservicearchitectuur.
    \item Fase 4 : Het opgestelde model toepassen op een hedendaags voorbeeld.
\end{itemize}

\subsection{fase 1}

In deze fase werd er een grondige literatuurstudie uitgevoerd worden. In deze studie werden verschillende aspecten van microservices en monolithische systemen behandeld. Het onderzoek leverde volgende sleutelwoorden op: \emph{holacracy, continuous delivery, domain-Driven design, serverless, API, REST, sockets, TCP, gateway, circuit breakers, load balancer en proxy}.

\subsection{fase 2}
Deze fase werd gebruikt om dieper in te gaan op de voordelen en de uitdagingen van microservices. Dit is een uitbreiding op fase 1.

\subsection{fase 3}
Het doel van deze fase was om de aspecten van de impact van de transformatie op te sommen. Vervolgens een template op te stellen die helpt bij het voorspellen van welke elementen van het bedrijf het meeste kans hebben om gemanifesteerd te worden tijdens de transformatie.

\subsection{fase 4}

In de laatste fase werden de ondervindingen uit de vorige fases toegepast op een huidige bedrijfssituatie. Door gebruik te maken van het vooropgestelde model, worden alle belangrijke elementen uitgelicht.






