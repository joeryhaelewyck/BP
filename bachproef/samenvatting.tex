%%=============================================================================
%% Samenvatting
%%=============================================================================

% TODO: De "abstract" of samenvatting is een kernachtige (~ 1 blz. voor een
% thesis) synthese van het document.
%
% Deze aspecten moeten zeker aan bod komen:
% - Context: waarom is dit werk belangrijk?
% - Nood: waarom moest dit onderzocht worden?
% - Taak: wat heb je precies gedaan?
% - Object: wat staat in dit document geschreven?
% - Resultaat: wat was het resultaat?
% - Conclusie: wat is/zijn de belangrijkste conclusie(s)?
% - Perspectief: blijven er nog vragen open die in de toekomst nog kunnen
%    onderzocht worden? Wat is een mogelijk vervolg voor jouw onderzoek?
%
% LET OP! Een samenvatting is GEEN voorwoord!

%%---------- Nederlandse samenvatting -----------------------------------------
%
% TODO: Als je je bachelorproef in het Engels schrijft, moet je eerst een
% Nederlandse samenvatting invoegen. Haal daarvoor onderstaande code uit
% commentaar.
% Wie zijn bachelorproef in het Nederlands schrijft, kan dit negeren, de inhoud
% wordt niet in het document ingevoegd.

\IfLanguageName{english}{%
\selectlanguage{dutch}
\chapter*{Samenvatting}



}{}

%%---------- Samenvatting -----------------------------------------------------
% De samenvatting in de hoofdtaal van het document

\chapter*{\IfLanguageName{dutch}{Samenvatting}{Abstract}}

De overgang van een monolithische architectuur naar een microservice-architectuur is een geuze onderneming. Veel bedrijven willen inspringen op de trend van microservices maar verliezen daarbij cruciale elementen uit het oog. Niet elk bedrijf is gebaat met een architectuur bestaande uit microservices, vooral kleinere ondernemingen hebben deze complexe vorm van architectuur niet nodig.
In dit onderzoek wordt er informatie verzameld omtrent de monolithische en microservice-architectuur. De voordelen en uitdagingen van deze architectuurvormen worden overlopen en in detail geanalyseerd. Vervolgens worden de twee systemen met elkaar vergeleken en wordt de focus gelegd op welke voordelen een transformatie naar microservices biedt. 

Een dergelijke transformatie wijzigt niet enkel de architectuur vorm, maar heeft ook impact op de dagelijkse werking van het bedrijf. Het personeel moet zich aanpassen, omgeschoold worden en sommige individuen verlaten de onderneming. De kosten zijn aanzienlijk. Hoe groter de monoliet, hoe meer tijd nodig is om de services los te koppelen van het systeem en om te zetten naar zelfstandige microservices. 

Er wordt een model opgesteld die helpt bij het aanduiden van de elementen die het meeste zullen veranderen door deze transformatie. De focus van het model ligt op het financiële en de sociale elementen van de onderneming.

Het model wordt toegepast op de situatie van TUI, een bedrijf die al enkele jaren bezig is met de transformatie van een monolithische architectuur naar microservices. Hieruit zal blijken dat een transformatie niet altijd het gewenste resultaat heeft. 

De conclusie is dat niet elke transformatie een succesverhaal zal zijn. Hoe beter het voorbereidend werk, hoe hoger de kans op slagen.

