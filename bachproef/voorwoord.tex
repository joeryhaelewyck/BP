%%=============================================================================
%% Voorwoord
%%=============================================================================

\chapter*{\IfLanguageName{dutch}{Woord vooraf}{Preface}}
\label{ch:voorwoord}

%% TODO:
%% Het voorwoord is het enige deel van de bachelorproef waar je vanuit je
%% eigen standpunt (``ik-vorm'') mag schrijven. Je kan hier bv. motiveren
%% waarom jij het onderwerp wil bespreken.
%% Vergeet ook niet te bedanken wie je geholpen/gesteund/... heeft

De structuur van microservices wint aan belang in het bedrijfsleven. Dit is ook het geval bij TUI.
Bij deze grootspeler in de toeristische sector werk ik ondertussen 6 jaar: de eerste 3 jaar als backend developer en tot heden als frontend developer. TUI is ook bezig met de overschakeling van een monolithische architectuur naar een systeem bestaande uit microservices.\\
Deze verandering loopt niet zo vlot als initieel gedacht. Dit bracht me tot het idee om een onderzoek uit te voeren naar welke impact deze omschakeling uiteindelijk met zich meebrengt. In theorie zijn er veel voordelen verbonden aan microservices, meer dan bij een monolithische structuur. In de praktijk daarentegen wegen de baten niet altijd door. Door dit onderzoek heb ik nu een beter beeld over hoe deze architecturen precies werken en kan ik deze kennis toepassen in het werkleven.\\  
Graag wil ik mijn co-promotors, \textbf{Yentl Verhelst} en \textbf{Alexander Sels}, bedanken voor hun begeleiding en ondersteuning. Vervolgens wil ik ook mijn promotor, \textbf{Giselle Vercauteren}, bedanken voor de opvolging en controle van mijn onderzoek. Tenslotte hoop ik dat ik de mensen die mijn werk nalezen iets kan bijleren over mijn onderwerp.
