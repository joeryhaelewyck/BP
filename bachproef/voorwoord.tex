%%=============================================================================
%% Voorwoord
%%=============================================================================

\chapter*{\IfLanguageName{dutch}{Woord vooraf}{Preface}}
\label{ch:voorwoord}

%% TODO:
%% Het voorwoord is het enige deel van de bachelorproef waar je vanuit je
%% eigen standpunt (``ik-vorm'') mag schrijven. Je kan hier bv. motiveren
%% waarom jij het onderwerp wil bespreken.
%% Vergeet ook niet te bedanken wie je geholpen/gesteund/... heeft

Microservices is een trend die de laatste jaren steeds meer in de kijker staat. Bedrijven beginnen volop deze architectuur te gebruiken. Het bedrijf TUI waar ik ondertussen al 6 jaar werk als developer, de eerste 3 jaar als backend developer en tot heden als frontend developer, is ook bezig met de overschakeling van een monolithische architectuur tot een systeem bestaande uit microservices.\\
Deze verandering loopt niet zo vlot als initieel gedacht. Dit bracht me tot het idee om een onderzoek uit te voeren naar welke impact deze omschakeling uiteindelijk met zich meebrengt. In theorie zijn er veel voordelen verbonden aan microservices, meer dan bij een monolithische structuur. Maar uit eigen ervaring weet ik dat dit niet altijd het geval is. Door dit onderzoek heb ik nu een beter beeld over hoe deze architecturen precies werken en kan ik deze kennis toe passen op mijn carrière.\\  
Graag wil ik mijn co-promotors, \textbf{Yentl Verhelst} en \textbf{Alexander Sels}, bedanken voor hun begeleiding en ondersteuning. Vervolgens wil ik ook mijn promotor, \textbf{Giselle Vercauteren}, bedanken voor de opvolging en controle van mijn onderzoek. Tenslotte hoop ik dat ik de mensen die mijn werk nalezen iets kan bijleren over mijn onderwerp.
